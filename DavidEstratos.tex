\documentclass{article}
\usepackage[utf8]{inputenc}

\title{CVR New Estimates - Stratification}
\author{Sulmont David}
\date{June 2019}

\begin{document}

\maketitle

\section{Stratification Strategy}

We stratified the country circumscriptions into the following groups: 

\begin{itemize}
  \item \texttt{AYA\_CENT} (``Central Ayacucho''): Provices of Huamanga, Cangallo, Vilcas-Huaman, Victor~Fajardo, Sucre, and Huanca~Sancos, in Ayacucho department. 

    Central Ayacucho was the first region of the country where the Shining Path initiated its armed insurrection. Along with North Ayacucho, this was the area of the country with the highest levels of casualties. At the beginning of the conflict, central Ayacucho was the main stronghold of the Shining Path and where this organization had heavily invested in political proselytism since the 1970's.

    We sub-divided this region into: 
    \begin{itemize} 
      \item \texttt{AYA\_CENT\_HUAMANGA\_AYA}:       District of Ayacucho (in Huamanga province).
      \item \texttt{AYA\_CENT\_CANGALLO\_CANGALLO}:  District of Ayacucho (in Cangallo province)..
      \item \texttt{AYA\_CENT\_VH\_VH}:              District of Vilcas~Huaman (in Vilcas~Huaman province).
      \item \texttt{AYA\_CENT\_HUAMANGA}: Rest of Huamanga province.
      \item \texttt{AYA\_CENT\_CANGALLO}: Rest of Cangallo province.
      \item \texttt{AYA\_CENT\_VH}: Rest of Vilcas~Huaman province.
      \item \texttt{AYA\_CENT\_VFSUCREHS}: Provinces of Victor~Fajardo, Sucre, and Huanca~Sancos.
    \end{itemize}

    This subdivision tries to distinguish the main urban centers from its most rural hinterland and the periphery, in order to capture the different dynamics of the conflict in those settings. The number of cases available in those areas facilitates this subdivision.  


  \item \texttt{AYA\_NORTE} (``North Ayacucho''): Provinces of Huanta and La Mar, in Ayacucho department.

    North Ayacucho was the first region where the Shining Path expanded its activities. It is the region where all the projects have recorded the highest number of victims. 

    We sub-divided this region into:
    \begin{itemize}
      \item \texttt{AYA\_NORTE\_CAPITALES}: Huanta district (in Huanta province), and San~Miguel district (in La~Mar province).
      \item \texttt{AYA\_NORTE\_HUANTA}: Rest of Huanta Province. 
      \item \texttt{AYA\_NORTE\_LAMAR}: Rest of La~Mar province.
    \end{itemize}

The subdivisions in this region follows the same logic as in the previous region

  \item \texttt{AYA\_NORTE\_CHUNGUI}: Chungui district (in Chungui province, Ayacucho department).

    Chungui in North Ayacucho, is the most affected district in the country during the armed conflict, in terms of the proportion of recorded and estimated number of victims in relation to its population. It is also the district where we have one of the highest levels of coverage in term of the variety of sources available. 

  \item \texttt{AYA\_SUR} (``South Ayacucho''): Provinces of Lucanas, Parinacochas, and Paucar~Del~Sara~Sara, in Ayacucho department.

    Those regions of Ayacucho were much less affected by the conflict than the rest of the department. So we can consider them peripheral areas, close to the "central" area of the country. 

 \item \texttt{SIERRA\_CENTRO} (``Central Highlands''): Departments of Pasco, Huancavelica, and the rest of Junin department.

    This region was the first region outside Ayacucho, where the Shining Path expanded its activities after the initial government crackdown against the Shining Path in Ayacucho between 1983-1984. 

    We sub-divided this region into:
    \begin{itemize}
      \item \texttt{SIERRA\_CENTRO\_PASCO}: Department of Pasco.
      \item \texttt{SIERRA\_CENTRO\_JUNIN}: Rest of department of Junin.
      \item \texttt{SIERRA\_CENTRO\_HUANCAVELICA}: department of Huancavelica.
    \end{itemize}
    
    This subdivision follows the main departments and provinces in the region affected with the conflict and where we can distinguish different dynamics and at the same time, have enough information to make them visible.
    
\item \texttt{SATIPO}: Satipo province in Junin department.

    Satipo is one of the most affected provinces in the central area or the country. In this area two of the insurgents groups (MRTA and Shining Path) had an important presence and particularly targeted and decimated the population of the Ashaninka ethnic group. 

 

  \item \texttt{NOR\_ORIENTE} (``Northeast''): Departments of San~Martin and Huanuco; provinces Coronel~Portillo and Padre~Abad from Ucayali department.

    Those regions were more affected by the conflict by the second half of the 1980's and at the beginning of the 1990's. The armed conflict in those regions was also closely intertwined with drug production (peasants producers of coca leaves and cocaine raw materials) and drug trafficking. 

    We sub-divided this region into:
    \begin{itemize}
      \item \texttt{NOR\_ORIENTE\_SAN\_MARTIN}: Department of San~Martin.
      \item \texttt{NOR\_ORIENTE\_HUANUCO}: Department of Huanuco.
      \item \texttt{NOR\_ORIENTE\_UCAYALI}: Provinces of Coronel~Portillo and Padre~Abad (in Ucayali department).
    \end{itemize}

    This subdivision follows the main departments and provinces in the region affected with the conflict and where we can distinguish different dynamics and at the same time, have enough information to make them visible.

  \item  \texttt{SIERRA\_SUR} (``South Highlands''): Departments of Cusco, Puno, and Apurimac.

As in the case of the Central Highlands, the Shining Path expanded its activities in those regions by the mid 1980's. However, unlike the Central Highlands, it was less successful in mobilizing some communities or in organizing armed activities.

    We sub-divided this region into:
    \begin{itemize}
      \item \texttt{SIERRA\_SUR\_CUSCO}: Department of Cusco.
      \item \texttt{SIERRA\_SUR\_PUNO}: Department of Puno.
      \item \texttt{SIERRA\_SUR\_APURIMAC}: Department of Apurimac.
    \end{itemize}

This subdivision follows the main departments and provinces in the region affected with the conflict and where we can distinguish different dynamics and at the same time, have enough information to make them visible.

  \item \texttt{LIMA\_CALLAO}: Province of Lima (department of Lima), and Callao Constitutional province.

The main urban center of the country is considered a stratum by itself, not only because it concentrated almost a third of the country's population, but because of the symbolic and political repercussions of the armed activities and casualties that took place in this region.

  \item \texttt{LIMA\_PROVINCIAS} (``Provinces of Lima''): Rest of department of Lima.

    Those provinces surrounded the main urban center of the country and they had a strategical value for subversive groups, which might explain the armed activity and pattern of victimization that took place in those areas. 

 \item \texttt{PERIFERIA} (``Surroundings''): Departments of Ica, Arequipa, Moquegua, Tacna, Ancash, La~Libertad, Cajamarca, Lambayeque, Piura, Tumbes, and Amazonas.

Those regions were some of the least affected by the conflict. Mainly along the coast and in the north of the country. Insurgent groups have much less roots and presence and capabilities to organize armed activities. 

  \item \texttt{SELVA} (``Amazon Region''): Departments of Loreto, Madre~De~Dios, and the rest of Ucayali department.

    Those regions were mainly peripheral areas of the conflict in the Amazon region, much less affected than the rest of the country.


\end{itemize}


\end{document}
